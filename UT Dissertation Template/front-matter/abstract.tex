\chapter*{Abstract}\label{ch:abstract}
The United States is currently looking at options for handling of spent nuclear fuel. Currently, there are $\approx 70,000$ metric tons of spent fuel in storage in the US alone. Pyroprocessing is a possible method for spent nuclear fuel reprocessing which was proven to work at Argonne National Laboratory. This masters thesis showcases a method for empirically modeling hybrid k-edge densitometry, one of the numerous possible safeguards needed for a reprocessing facility. This is accomplished via 54 MCNP 2 stage KED simulations, as well as 54 2-stage XRF simulations, for a total of 216 MCNP simulations. These MCNP simulation outputs are then imported into Python, for empirical fitting with the SciPy module of Python. The end results are empirical functions for the magnitude of the k-edge drop of uranium and plutonium, as well as empirical functions for the XRF peaks for uranium and plutonium.

The semi-empirical functions are then implemented into the Sandia National Laboratory Separation and Safeguards Performance Model EChem (SSPM Echem) Simulink model. The empirical functions as well as multiple representative figures and tables are presented, showcasing the ability of the Simulink module to correctly predict the KED drops as well as the XRF peaks. Several divergence scenarios are evaluated and results shown.